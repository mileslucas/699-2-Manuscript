\section{Introduction}\label{sec:intro}

The last twenty years of astronomy have seen a revolution in planetary science, with thousands of exoplanets discovered around nearby stars. While our understanding of exoplanet demographics has lept forward in recent years, fundamental questions remain. What are the dominant planet formation pathways? How do planetary atmospheres form and evolve? Is there life on other planets?

These questions can only be answered through the spectroscopic characterization of exoplanets over a range of masses and orbital separations from their host stars. While each method of exoplanet detection plays an important role in surveying planets, only the methods of transits and direct imaging allow for the possibility of spectroscopic characterization. Directly imaging exoplanets is versatile- the planet's orbit does not have to transit its host star and the orbital period is less crucial for detection compared to radial velocity, transit, and astrometric methods \citep{2013pss3.book..489W}. The challenge in direct imaging, however, is appropriately separating and nulling the starlight to detect the dim exoplanet signal \citep{2010exop.book..111T}.

The field of direct imaging (or high-contrast imaging) comprises modern instrumentation, observational techniques, and post-processing algorithms to overcome the obstacles in imaging a planet outside our solar system. Adaptive optics (AO) play a critical role in direct imaging by stabilizing the stellar point-spread function (PSF), enabling modern coronagraphs to attenuate the stellar signal \citep{2005ApJ...629..592G}. AO systems consist of a wavefront sensor, deformable mirror, and a real-time controller. The controller takes the images from the wavefront sensor and calculates the deformable mirror commands. This loop needs to run at kHz speeds to enable high-contrast imaging \citep{2018ARA&A..56..315G}.

% Typically, the AO system is ``upstream'' of the science instrument, which means that any wavefront errors ``downstream'' of the wavefront sensor cannot be corrected (e.g., a polishing defect). These wavefront errors are referred to as non-common path aberrations. The aberrations accumulate in the focal plane as quasi-static speckles, which are aberrations of the true PSF \citep{2018ARA&A..56..315G}. The speckles have structure, but it is variable due to slow-changing effects such as thermal gradients. This variability leads to their quasi-static nature.

AO systems are constrained by engineering and runtime factors such as the density of deformable mirror segments, and the speed of the real-time control. These factors limit how well an AO system can correct a wavefront. For ground-based telescopes these limits are significant due to high-speed atmospheric turbulence variation, which evolves faster than the AO system can correct, leading to residual wavefront errors \citep{Soummer_2007}. These errors manifest in fast speckles with coherence times $\sim$0.01s \citep{2018ARA&A..56..315G}, which is much less than a typical science integration ($\sim$10-100s). This causes the speckles to average out over an integration and form a halo with exponentially decreasing brightness from its center. The quasi-static speckles and halo are hard to distinguish from astrophysical signals, limiting our sensitivity to exoplanets (see \autoref{fig:wdh}).

\begin{figure}
    \centering
    % \epsscale{0.7}
    \plotone{wdh}
    \caption{Coronagraphic focal-plane images showing the wind-driven halo effect. The left figure is a simulated model of the SPHERE-IRDIS instrument with a wind-driven halo. The right figure is an H2-band exposure from SPHERE-IRDIS. The circled regions in the right figure are known optical artifacts. The existence of the wind-driven halo greatly compromises the sensitivity for detecting exoplanets in high-contrast imaging. Adapted from \cite{2020AA...638A..98C}.}
    \label{fig:wdh}
\end{figure}
A na\"ive solution to overcome the time-delay would be to run the AO loop at a faster rate, but there is a speed limit due to photon noise on the wavefront sensor when the integration time is too short. For example, on the Subaru telescope with its \SI{8.2}{\meter} aperture, a magnitude 8 star produces approximately \SI{6.9e4}{\photon\per\second} in H-band divided across $\sim$\num{7200} subapertures in the wavefront sensor, which means at \SI{2}{\kilo\hertz} loop rate only 8 photons arrive in each subaperture of each frame. This relationship between loop rate and photon noise is particularly challenging for traditional integrator control laws (the math that determines the deformable mirror commands given the wavefront sensor measurement). These laws use an average of the previous few wavefront measurements to help attenuate the effects of photon noise. This time-averaging, though, exacerbates the time-delay problem and must be carefully balanced for each target. Besides, there will always be the finite time delay of communications and computations within the control system along with mechanical restrictions from the deformable mirror(s). This means there is a fundamental limit to improving performance of wind-driven effects using traditional AO control laws.

\subsection{Predictive wavefront control}\label{sec:pwfc}

Predictive wavefront control improves on traditional AO control laws by \textit{predicting} the future wavefront errors. Predicting the future wavefront would allow extreme AO systems to avoid the time-lag error which creates the wind-driven halo in the focal plane images \citep{2018ARA&A..56..315G}. By forming this prediction off an ensemble of frames in the past, there is also the benefit of reduced photon noise, which means predictive control can enable imaging of fainter targets, too. This is the formulation behind the \textit{Empirical Orthogonal Functions} (EOF; \citealp{guyon_adaptive_2017}) predictive wavefront control law, which uses a linear combination of previous wavefront measurements (which have been projected onto an orthogonal subspace) to predict the future wavefront. Using a linear control law is theoretically well-suited to counteract the effects of wind in the atmosphere, since we believe wind is comprised of a linear combination of flows with constant direction and speed (the \textit{frozen-flow hypothesis}). \citet{guyon_adaptive_2017} reports an expected $\sim$100 times improvement in root-mean-square (RMS) wavefront error using EOF, which leads to $\sim$3 orders of magnitude improved contrast for detecting faint exoplanets. On-sky performance, however, is only a factor of a few improved and in certain cases degenerate non-linear effects overwhelm the algorithm.

\begin{figure}
    \centering
    \epsscale{0.8}
    \plotone{eof-performance}
    \caption{Demonstration of predictive control for high-contrast imaging. The following panels show simulated focal-plane PSF: (a) raw PSF contrast (scale divided by the maximum value), (b) contrast with coronagraphic corrections; here the fast-evolving speckle halo is clear, and (c) contrast with predictive control and coronagraphic corrections. Imaging a Jupiter analogue requires contrast of $\sim10^{-8}$ and for an Earth analogue in reflected light is $\sim10^{-10}$, for reference.}
    \label{fig:eof-perf}
\end{figure}

The shortcoming of predictive control on-sky is an active topic of adaptive optics research \citep{2018ARA&A..56..315G} and is paramount to the success of large-aperture ($>$\SI{10}{\meter}) telescopes for exoplanet imaging. The Thirty Meter Telescope's Planetary Science Imager (PSI) require predictive control to attenuate the time-lag error by at least an order of magnitude to reach their goals of imaging and taking spectra of rocky exoplanets around M-type stars \citep{2018SPIE10703E..0ZG}. This is a parameter space for exoplanets which has not yet been explored by indirect methods and is well-suited for discovering habitable terrestrial planets. Why do the on-sky tests perform so much worse than the testbed results? Is the frozen-flow hypothesis valid in the regime of extreme AO? What are the relevant time scales for the dynamics driving fast speckles? These questions are key to answer in the coming years in order to push ground-based astronomy further in exoplanet research.

\subsection{Studying wind speed}\label{sec:windspeed}

Wind speed is important to study in order to improve current predictive control methods. Alluded to in \autoref{sec:pwfc}, the frozen-flow hypothesis of wind is not guaranteed to be true, and discrepancies may be part of why predictive control does not perform as well on-sky as in theory. In addition, part of the EOF control algorithm requires training a low-rank orthogonal basis to wavefront data. This decomposition is great for filtering sources of wavefront errors like mechanical vibrations since they follow a regular pattern that can be encoded into the subspace. The constant speed and directions of the flows in the frozen-flow hypothesis are likely to be encoded into this subspace, which should improve the prediction of wind-related effects! However, if the flows evolve, by changing direction or speed, this leads to a dramatic decrease in the performance of the predictor. In current applications the EOF predictors are retrained every 30 minutes to alleviate the problems, but a more thorough study of the dynamics of wind speed (how much it changes, and how rapidly) would greatly optimize the application and performance of predictive control. In fact, developing an \textit{online} algorithm that can be run alongside the EOF control law to estimate when to retrain would greatly improve the stability of the wavefront control.

Although it seems obvious to reference atmospheric and meteorological data to determine how wind speed evolves, we instead choose to use telemetry from the AO system itself. While the dynamics of wind evolution are itself an interesting research topic, the concern is with how wind affects the incoming wavefronts. The data-driven approach of using AO telemetry is much more flexible for this scenario than a weather tower, and allows probing the entire atmosphere above the telescope, which would normally require a controlled weather balloon experiment. The rest of this report details how we can use AO telemetry to measure wind speed (\autoref{sec:methods}), including the details of our algorithm (\autoref{sec:algo}) for application on the Subaru Coronagraphic Extreme Adaptive Optics instrument (SCExAO; \citealp{guyon_wavefront_2011}), the verification of the algorithm in simulated and testbed experiments (\autoref{sec:simulated}, \autoref{sec:turbulence}), and the current status of on-sky application of the algorithm (\autoref{sec:onsky}).