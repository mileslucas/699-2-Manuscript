\section{Introduction} \label{sec:intro}

The last twenty years of astronomy have seen a revolution in planetary science,with thousands of exoplanets discovered around nearby stars. While our understanding of exoplanet demographics has leapt forward in recent years, fundamental questions remain. What are the dominant planet formation pathways? How do planetary atmospheres form and evolve? Is there life on other planets?

These questions can only be answered through the spectroscopic characterization of exoplanets over a range of masses and orbital separations from their host stars. While each method of exoplanet detection plays an importantrole in surveying planets, only the methods of transits and direct imaging allow for the possibility of spectroscopic characterization. Directly imaging exoplanets is versatile- the planet's orbit does not have to transit its host star and the orbital period is less crucial for detection compared to radial velocity, transit, and astrometric methods \citep{2013pss3.book..489W}. The challenge in direct imaging, however, is appropriately separating and nulling the starlight to detect the dim exoplanet signal \citep{2010exop.book..111T}.

The field of direct imaging (or high-contrast imaging) comprises modern instrumentation, observational techniques, and post-processing algorithms to overcome the obstacles in imaging a planet outside our solar system. Adaptive optics (AO) play a critical role in direct imaging by stabilizing the stellar point-spread function (PSF), enabling modern coronagraphs to attenuate the stellar signal \citep{2005ApJ...629..592G}. AO systems consist of a wavefront sensor, deformable mirror, and a real-time controller. The controller takes the images from the wavefront sensor and calculates the deformable mirror commands. This loop needs to run at kHz speeds to enable high-contrast imaging \citep{2018ARA&A..56..315G}.

Typically, the AO system is ``upstream" of the science instrument, which means that any wavefront errors ``downstream" of the wavefront sensor cannot be corrected (e.g., a polishing defect). These wavefront errors are referred to as non-common path aberrations. The aberrations accumulate in the focal plane as quasi-static speckles, which are aberrations of the true PSF \citep{2018ARA&A..56..315G}. The speckles have structure, but it is variable due to slow-changing effects such as thermal gradients. This variability leads to their quasi-static nature.

AO systems are also constrained by engineering and runtime factors such as the density of deformable mirror segments, and the speed of the real-time control. These factors limit how well an AO system can correct a wavefront. For ground-based telescopes these limits are significant due to high-speed atmospheric turbulence variation, which evolves faster than the AO system can correct, leading to residual wavefront errors \citep{Soummer_2007}. These errors manifest in fast speckles with coherence times $\sim$0.01s, which is much less than a typical science integration ($\sim$10-100s). This causes the speckles to average out over an integration and form a halo with exponentially decreasing brightness from its center. The quasi-static speckles and halo are hard to distinguish from astrophysical signals, limiting our sensitivity to exoplanets (see~\ref{fig:wdh}).

\begin{figure}
    \centering
    % \epsscale{0.7}
    \plotone{wdh}
    \caption{Coronagraphic focal-plane images showing the wind-driven halo effect. The left figure is a simulated model of the SPHERE-IRDIS instrument with a wind-driven halo. The right figure is an H2-band exposure from SPHERE-IRDIS. The circled regions in the right figure are known optical artifacts. The existence of the wind-driven halo greatly compromises the sensitivity for detecting exoplanets in high-contrast imaging. Adapted from \cite{2020AA...638A..98C}.}
    \label{fig:wdh}
\end{figure}

\subsection{Predictive wavefront control} \label{sec:pwfc}

Predictive wavefront control improves on current AO control laws (the math that determines the DM commands given the wavefront sensor measurement) by \textit{predicting} the future wavefront errors. Predicting the future wavefront would allow extreme AO systems to avoid the time-lag error which creates the wind-driven halo in the focal plane images.  

\subsection{Studying wind speed} \label{sec:windspeed}

