\section{Conclusions}\label{sec:conclusions}

Improving predictive control methods for extreme AO is a necessary and prescient problem as ground-based telescope apretures grow larger and the push to discover Earth-like exoplanets increases. In this report, we outlined the shortcomings of predictive control and specifically how wind negatively affects the performance of high-contrast imaging. We designed a study to improve our knowledge of the dynamics of how wind evolves in the pupil plane. We created an algorithm for estimating wind speed using pseudo-open-loop data and validated it on simulated data and on the SCExAO testbed. This algorithm is not yet suited for on-sky data, which suffer from many more complications and are generally noisier than injecting simulated data on the testbed. Our results only recover the apparent motion in on-sky data with $\sim$20\% accuracy. Before we can try to analyze wind dynamics we need to find appropriate pre-processing steps to maximize the signal of the motion in the cross-correlation. We also need to validate that the pre-processing is applicable to different data with different apparent motion, otherwise we may bias the speed measurements implicitly.

One of the explanations for the poor on-sky performance is a consequence of the Subaru telescope architecture: the AO188 system has already filtered the incident wavefront before entering SCExAO, which could attenuate the wind-driven effects (especially the ground-layer turbulence), and unfortunately SCExAO cannot interface with the AO188 telemetry, currently. Another group has been analyzing wind speed using the University of Hawai'i 88-inch telescope's 'Imaka ground-layer AO system \citep{2016SPIE.9909E..02C}. 'Imaka is a much different system than SCExAO: it is a multi-conjugate system with wide-field coverage using 5 Shack-Hartmann wavefront sensors rather than a single pyramid wavefront sensor. Nonetheless, this group has combined a 3-dimensional (time, x, and y) cross-correlation technique with Fourier analysis to retrieve individual wind flows (McEwen et al., in prep.). Their cross-correlation technique adds a step which reduces the effects of pixel-to-pixel noise by averaging many cross-correlation maps together, which we could similarly apply to our algorithm.

The Fourier techniques used by this group are based on the slopes of the Shack-Hartmann sensor and therefore are not directly applicable to our analysis with SCExAO's pyramid wavefront sensor. However, SCExAO uses Zernike polynomials as a modal basis. This means rather than analyzing wavefront data in full frame reconstructions, we can represent it as a vector of Zernike coefficients. \citet{1993JOSAA..10..957R} derives power spectra for Zernike modes and uses them to derive turbulence layers similar to the McEwen algorithm \citep[see][fig.~14]{1993JOSAA..10..957R}. In future work, we would like to explore a Fourier analysis of the Zernike modes SCExAO uses.